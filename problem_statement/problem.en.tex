\problemname{The Minimized Maximum Height Difference}
In a kindergarten, the teacher wants the children to dance in a ring, so she asks the children to gather round, hold hands, and form a circle. But the thing is there are n different children with heights h, from h[1], h[2], ..., h[n]. She hopes the children can dance orderly, so she expects the height difference between each pair of children standing next to each other is minimized. In other words, she wants to minimize the maximum of height difference among all neighboring children. For instance, there are 6 children with the heights of 1, 1, 2, 2, 3, 3. One way to arrange children in a circle is: 1, 2, 3, 3, 2, 1. In this case, the minimized maximum of height difference among all neighboring children is 1. Can you help the teacher to find a way to do so? If so, please help her find the minimized maximum of height difference among all neighboring children M.

\section*{Input}
The first line of the input contains an integer T giving the number of test cases, 1 $\leq$ T $\leq$ 100. The following 2T lines specify the test cases, two lines per case: the first line gives a positive integer 2 $\leq$ n $\leq$ 100 which is the number of children standing in a circle, and the following line contains the n different positive integer heights: h[1], h[2], ..., h[n]. For any scenario, the height is no greater than 10,000.

\section*{Output}
For each input case a single line should be output. The output should be a positive integer M that is the minimized maximum of height difference between each pair of children standing next to each other.